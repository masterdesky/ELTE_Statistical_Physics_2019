\section*{Problem 3.}
\subsection*{Question}
For non-interacting fermions one can define a characteristic temperature $T_{\text{deg}}$ by that temperature where the chemical potential is equal to zero:

\begin{equation*}
\mu \left( T = T_{\text{deg}} \right) = 0.
\end{equation*}
By dimensional analysis

\begin{equation*}
k_{B} T_{\text{deg}}
=
z \frac{\hbar^{2}}{2m} \left( \frac{N}{V} \right)^{2/3}
\end{equation*}
where $z$ is a dimensionless number. Calculate this number $z$ exactly and numerically.

\subsection*{Solution}
At hight temperature the chemical potential is negative, but for lower temperature its sign changes. There will be a certain $T_{\text{deg}}$ temperature, where it is zero. It could be concluded \citep{Lee1990}, that for non-interacting fermions, this characteristic temperature could be expressed as follows:

\begin{equation} \label{eq:16}
\Gamma \left( 1 + D/2 \right) * \left( \frac{\mu_{0}}{k_{B} T_{\text{deg}}} \right)^{D/2}
=
\left( 1 - 2^{1 - D/2} \right) \zeta \left( D/2 \right)
\end{equation}
Where $\Gamma$ is the gamma function, $\zeta$ is the Riemann zeta function, $D$ is the dimension number (here, $D := 3$) and $\mu_{0}$ is the chemical potential at $T=0$. For $D=3$ the above equation could be rephrased in the following way:
\begin{equation} \label{eq:17}
\Gamma \left( 5/2 \right) * \left( \frac{\mu_{0}}{k_{B} T_{\text{deg}}} \right)^{3/2}
=
\left( 1 - 2^{- 1/2} \right) \zeta \left( 3/2 \right)
\end{equation}
\begin{equation} \label{eq:18}
\frac{3 \sqrt{\pi}}{4} * \left( \frac{\mu_{0}}{k_{B} T_{\text{deg}}} \right)^{3/2}
=
\left( 1 - \frac{1}{\sqrt{2}} \right) \left( \frac{2}{\sqrt{\pi}} \int_{0}^{\infty} \frac{\sqrt{t}}{e^{t} - 1}\,dt \right)
\end{equation}
\begin{equation} \label{eq:19}
\left( \frac{\mu_{0}}{k_{B} T_{\text{deg}}} \right)^{3/2}
=
\underbrace{
\frac{8}{3 \pi} \left( 1 - \frac{1}{\sqrt{2}} \right) \left( \int_{0}^{\infty} \frac{\sqrt{t}}{e^{t} - 1}\,dt \right)
}_{> 0}
\end{equation}
\begin{equation} \label{eq:20}
\frac{\mu_{0}}{k_{B} T_{\text{deg}}}
=
\frac{1}{
\left(
\frac{8}{3 \pi} \left( 1 - \frac{1}{\sqrt{2}} \right) \left( \int_{0}^{\infty} \frac{\sqrt{t}}{e^{t} - 1}\,dt \right)
\right)^{2/3}
}
\end{equation}
\begin{equation} \label{eq:21}
T_{\text{deg}}
=
\frac{\mu_{0}}{k_{B}}
\left(
\frac{8}{3 \pi} \left( 1 - \frac{1}{\sqrt{2}} \right) \left( \int_{0}^{\infty} \frac{\sqrt{t}}{e^{t} - 1}\,dt \right)
\right)^{2/3}
\end{equation}
The value of $z$ could be expressed by simple reordering of the equation in the task description:

\begin{equation} \label{eq:22}
z
=
\frac{k_{B} T_{\text{deg}}}{\frac{\hbar^{2}}{2m} \left( \frac{N}{V} \right)^{2/3}}
=
\frac{k_{B} T_{\text{deg}} 2m}{\hbar^{2}}
\left( \frac{N}{V} \right)^{-2/3}
\end{equation}
Substituting the derived formula for $T_{\text{deg}}$, the $k_{B}$ factor cancels out:

\begin{equation} \label{eq:23}
z
=
\frac{2m \mu_{0}}{\hbar^{2}}
\left(
\frac{8}{3 \pi} \left( 1 - \frac{1}{\sqrt{2}} \right) \left( \int_{0}^{\infty} \frac{\sqrt{t}}{e^{t} - 1}\,dt \right)
\right)^{2/3}
\left( \frac{N}{V} \right)^{-2/3}
\end{equation}
At zero temperature the chemical potential is equals to the Fermi energy ($E_{F}$), which could be expressed for non-interacting half-integer spin particles as follows \citep{Glyde2014}:
\begin{equation}
\mu_{0}
=
E_{F}
=
\frac{\hbar^{2}}{2m} \left( 3 \pi^{2} \right)^{2/3} \left( \frac{N}{V} \right)^{2/3}
\end{equation}
Substituting back to the previous equation, we get the form

\begin{equation}
z
=
\frac{2m}{\hbar^{2}} \left( \frac{N}{V} \right)^{-2/3} * \frac{\hbar^{2}}{2m} \left( 3 \pi^{2} \right)^{2/3} \left( \frac{N}{V} \right)^{2/3}
\left(
\frac{8}{3 \pi} \left( 1 - \frac{1}{\sqrt{2}} \right) \left( \int_{0}^{\infty} \frac{\sqrt{t}}{e^{t} - 1}\,dt \right)
\right)^{2/3}
\end{equation}
\begin{equation}
z
=
\left( 3 \pi^{2} \right)^{2/3}
\left(
\frac{8}{3 \pi} \left( 1 - \frac{1}{\sqrt{2}} \right) \left( \int_{0}^{\infty} \frac{\sqrt{t}}{e^{t} - 1}\,dt \right)
\right)^{2/3}
\end{equation}
Which is an exact value. It could be approximated

\begin{equation}
z
\approx
6.62246...
\end{equation}