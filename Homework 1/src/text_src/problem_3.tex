\section*{Problem 3.}
\subsection*{Question}
Show that for a pure n-particle fermionic state (given by a single Slater-determinant in first quantization)

\begin{equation}
P \left( \boldsymbol{r}, s, \boldsymbol{r'}, s' \right)
=
\varrho \left( \boldsymbol{r}, s \right) * \varrho \left( \boldsymbol{r'}, s' \right)
-
\left| \varrho \left( \boldsymbol{r}, s, \boldsymbol{r'}, s' \right) \right|^{2}
\end{equation}
where $\varrho \left( \boldsymbol{r}, s \right)$ is the spin dependent density and $\varrho \left( \boldsymbol{r}, s, \boldsymbol{r'}, s' \right)$ is the density matrix.

\subsection*{Solution}
On the course we've seen, that the pair distribution function is the following:

\begin{equation}
P \left( \boldsymbol{r}, s, \boldsymbol{r'}, s' \right)
=
\Psi^{\dagger} \left( \boldsymbol{r}, s \right)
\Psi^{\dagger} \left( \boldsymbol{r'}, s' \right)
\Psi \left( \boldsymbol{r}, s \right)
\Psi \left( \boldsymbol{r'}, s' \right)
\end{equation}
We also derived, the spin-dependent density operator is the following:

\begin{equation}
\varrho \left( \boldsymbol{r}, s \right)
=
\Psi^{\dagger} \left( \boldsymbol{r}, s \right)
\Psi \left( \boldsymbol{r}, s \right)
\end{equation}
\begin{equation}
\varrho \left( \boldsymbol{r'}, s' \right)
=
\Psi^{\dagger} \left( \boldsymbol{r'}, s' \right)
\Psi \left( \boldsymbol{r'}, s' \right)
\end{equation}
It is also the diagonal of the density matrix which could be written in the following form:

\begin{equation}
\varrho \left( \boldsymbol{r}, s, \boldsymbol{r'}, s' \right)
=
N \sum_{s_{2}} \dots \sum_{s_{n}}
\int d^{3} r_{2} \dots d^{3} r_{n}
\Psi^{\ast} \left( \boldsymbol{r_{1}}, s_{1}, \dots \right) * \Psi \left( \boldsymbol{r_{1}}, s_{1}, \dots \right)
\end{equation}
\begin{equation}
\text{diag} \left[ \varrho \left( \boldsymbol{r}, s, \boldsymbol{r'}, s' \right) \right]
=
\varrho \left( \boldsymbol{r}, s, \boldsymbol{r'} = \boldsymbol{r}, s' = s \right)
=
\varrho \left( \boldsymbol{r}, s \right)
\end{equation}
The absolute value of a positive operator, like $\varrho \left( \boldsymbol{r}, s, \boldsymbol{r'}, s' \right)$ is well-defined and is the following:

\begin{equation}
\left| \varrho \left( \boldsymbol{r}, s, \boldsymbol{r'}, s' \right) \right|
=
\sqrt{\varrho \left( \boldsymbol{r}, s, \boldsymbol{r'}, s' \right) * \varrho \left( \boldsymbol{r}, s, \boldsymbol{r'}, s' \right)}
\end{equation}
which positive square root exist.