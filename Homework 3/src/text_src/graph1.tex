\subsection*{Graph 1. \moment}

\begin{center}
\begin{tikzpicture}
  \begin{feynman}
    \vertex[empty dot, label=below:{$x$}] (a1){};
    \vertex[dot, right=2cm of a1, label=below:{$x_{1}$}] (a2){};
    \vertex[dot, right=2cm of a2, label=above:{$x_{2}'$}] (a3){};
    \vertex[dot, right=2cm of a3, label=above:{$x_{2}$}] (a4){};
    \vertex[empty dot, right=2cm of a4, label=below:{$x'$}] (a5){};
    
    \vertex[dot, above=1.5cm of a2, label=above:{$x_{1}'$}] (b1){};
    \vertex[above=1.5cm of b1] (b2);
    
    \vertex[above=0.75cm of b1] (c1);
    \vertex[right=1cm of a3] (c2);

    \diagram* {
      {[edges=fermion]
        (a5) -- [momentum=$\boldsymbol{k}; i \nu_{n}$] (a4),
        (a4) -- [momentum'=$\boldsymbol{q}; i \nu_{m}$] (a3),
        (a3) -- [momentum=$\boldsymbol{k}; i \nu_{n}$] (a2),
        (a2) -- [momentum=$\boldsymbol{k}; i \nu_{n}$] (a1),
      },
      (a2) -- [rmomentum=$\boldsymbol{0}; 0$, dash pattern=on 4pt off 4pt] (b1),
      (a4) -- [rmomentum'=$\boldsymbol{q}-\boldsymbol{k}$, edge label=$i \nu_{m} - i \nu_{n}$, half left, dash pattern=on 4pt off 4pt] (a3),
      (b1) -- [half right, looseness=1.65, in=-95, out=-85] (b2),
      (b2) -- [momentum'=$\boldsymbol{q'}; i \nu_{m}'$, half right, looseness=1.65, in=-95, out=-85, fermion] (b1),
    };
    %\centerarc[fermion](c1)(0:360:0.75)

  \end{feynman}
\end{tikzpicture}
\end{center}

\subsection*{Solution}
Fermion propagator lines are building connections between the points $\left( x' \to x_{2} \right)$, $\left( x_{2} \to x_{2}' \right)$, $\left( x_{2}' \to x_{1} \right)$, $\left( x_{1} \to x \right)$ and $\left( x_{1}' \to x_{1}' \right)$, where the last one is a fermion loop. Their contributions are:

\begin{equation} \label{eq:1}
- \mathcal{G}_{0} \left( \boldsymbol{k}; i \nu_{n} \right)
=
- \frac{1}{i \nu_{n} - \frac{1}{\hbar} \left( \varepsilon_{\boldsymbol{k}} - \mu \right)}
\end{equation}
Using this relation, the contribution of fermion propagator lines in this graph are the following:

\begin{align} \label{eq:2}
\mathcal{G}_{F}
&=
\left[ - \mathcal{G}_{0} \left( \boldsymbol{k}; i \nu_{n} \right) \right]
*
\left[ - \mathcal{G}_{0} \left( \boldsymbol{q}; i \nu_{m} \right) \right]
*
\left[ - \mathcal{G}_{0} \left( \boldsymbol{k}; i \nu_{n} \right) \right]
*
\left[ - \mathcal{G}_{0} \left( \boldsymbol{k}; i \nu_{n} \right) \right]
*
\left[ - \mathcal{G}_{0} \left( \boldsymbol{q}'; i \nu_{m}' \right) \right]
= \nonumber \\
&=
- \left[ \mathcal{G}_{0} \left( \boldsymbol{k}; i \nu_{n} \right) \right]^{3}
*
\mathcal{G}_{0} \left( \boldsymbol{q}; i \nu_{m} \right)
*
\mathcal{G}_{0} \left( \boldsymbol{q}'; i \nu_{m}' \right)
\end{align}
Which could be expressed as follows according to Eq. (\ref{eq:1}):

\begin{equation} \label{eq:3}
\mathcal{G}_{F}
=
- \left[ \frac{1}{i \nu_{n} - \frac{1}{\hbar} \left( \varepsilon_{\boldsymbol{k}} - \mu \right)} \right]^{3}
*
\frac{1}{i \nu_{m} - \frac{1}{\hbar} \left( \varepsilon_{\boldsymbol{q}} - \mu \right)}
*
\frac{1}{i \nu_{m}' - \frac{1}{\hbar} \left( \varepsilon_{\boldsymbol{q}'} - \mu \right)}
\end{equation}
All contribution from $\mathcal{G}_{0} \left( \boldsymbol{k}; i \nu_{n} \right)$ propagators should be multiplied by the factor $e^{i \nu_{n} \eta}$, if the propagator line is a closed loop itself, or whether its endpoints are connected with an interaction line. Here both $\mathcal{G}_{0} \left( \boldsymbol{q}; i \nu_{m} \right)$ and $\mathcal{G}_{0} \left( \boldsymbol{q}'; i \nu_{m}' \right)$ are subjects to this condition, and should be multiplied by the previously mentioned factor. Thus the contribution from the propagator lines are changing as the following:

\begin{equation} \label{eq:4}
\mathcal{G}_{F}
=
- \left[ \frac{1}{i \nu_{n} - \frac{1}{\hbar} \left( \varepsilon_{\boldsymbol{k}} - \mu \right)} \right]^{3}
*
\frac{e^{i \nu_{m} \eta}}{i \nu_{m} - \frac{1}{\hbar} \left( \varepsilon_{\boldsymbol{q}} - \mu \right)}
*
\frac{e^{i \nu_{m}' \eta}}{i \nu_{m}' - \frac{1}{\hbar} \left( \varepsilon_{\boldsymbol{q}'} - \mu \right)}
\end{equation}
Interaction lines running between the points $\left( x_{1}, x_{1}' \right)$ and $\left( x_{2}, x_{2}' \right)$. The contribution from an ($\boldsymbol{k}; i \nu_{n}$) interaction line is frequency-independent and is the following:

\begin{equation} \label{eq:5}
\mathcal{G}_{I} \left( \boldsymbol{k}; i \nu_{n} \right)
=
- \frac{1}{\hbar} v \left( \boldsymbol{k} \right)
\end{equation}
Using this relation the total contribution of interaction lines in this graph is

\begin{equation} \label{eq:6}
\mathcal{G}_{I}
=
\left[ - \frac{1}{\hbar} v \left( \boldsymbol{0} \right) \right]
*
\left[ - \frac{1}{\hbar} v \left( \boldsymbol{q} - \boldsymbol{k} \right) \right]
=
\frac{1}{\hbar^{2}} v \left( \boldsymbol{0} \right) v \left( \boldsymbol{q} - \boldsymbol{k} \right)
\end{equation}
To put everything together, we need to sum over all independent frequencies and momentums:

\begin{equation} \label{eq:7}
\mathcal{G} \left( \boldsymbol{k}; i \nu_{n} \right)
=
\frac{1}{\beta \hbar} \sum_{i \nu_{n}}
\frac{1}{\beta \hbar} \sum_{i \nu_{m}}
\frac{1}{\beta \hbar} \sum_{i \nu_{m}'}
\frac{1}{V} \sum_{\boldsymbol{k}}
\frac{1}{V} \sum_{\boldsymbol{q}}
\frac{1}{V} \sum_{\boldsymbol{q}'}
\mathcal{G}_{F} * \mathcal{G}_{I}
=
\frac{1}{\beta^{3} \hbar^{3}} \frac{1}{V^{3}}
\sum_{i \nu_{n}}
\sum_{i \nu_{m}}
\sum_{i \nu_{m}'}
\sum_{\boldsymbol{k}}
\sum_{\boldsymbol{q}}
\sum_{\boldsymbol{q}'}
\mathcal{G}_{F} * \mathcal{G}_{I}
\end{equation}
Finally this contribution should be multiplied by another factor, which is the contribution of the propagator loops itself:

\begin{equation} \label{eq:8}
\mathcal{G}_{L}
=
\left[ \pm \left( 2s + 1 \right) \right]^{L}
\end{equation}
Where $L$ is the number of propagator loops. Since there are only one propagator loop, $L = 1$. Thus the final Green's function is the following:

\begin{equation} \label{eq:9}
\mathcal{G} \left( \boldsymbol{k}; i \nu_{n} \right)
=
\left[ \pm \left( 2s + 1 \right) \right]^{1} *
\frac{1}{\beta^{3} \hbar^{3}} \frac{1}{V^{3}}
\sum_{i \nu_{n}}
\sum_{i \nu_{m}}
\sum_{i \nu_{m}'}
\sum_{\boldsymbol{k}}
\sum_{\boldsymbol{q}}
\sum_{\boldsymbol{q}'}
\mathcal{G}_{F} * \mathcal{G}_{I}
\end{equation}