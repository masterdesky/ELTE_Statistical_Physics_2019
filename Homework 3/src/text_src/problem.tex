\section*{Question}
\textit{(On the next page there is a table.)} Next to your name you can find five numbers under the column $\mathcal{G} \left( X, X' \right)$. Give the contribution of those graphs to $\mathcal{G} \left( X, X' \right)$ in coordinate space. Choose appropriate coordinates at the vertices and write it on your figures in the solutions. Next to your name you can find another five numbers under the column $\mathcal{G} \left( \boldsymbol{k}, i \omega_{n} \right)$. Give the contribution of those graphs to $\mathcal{G} \left( \boldsymbol{k}, i \omega_{n} \right)$. Once again, use clear notations for your conventions together with a picture showing the newly introduced momenta and frequencies. Classify your graphs if they are reducible or irreducible. \\ \\
The row of the table with my name:
\begin{center}
\begin{tabular}{|r|c|c|c|c|c|c||c|c|c|c|c|}
\hline
No.    & Name       & \multicolumn{5}{c||}{$G \left( X, X' \right)$} & \multicolumn{5}{c|}{$G \left( \boldsymbol{k}, i \omega_{n} \right)$} \\
\hline
\vdots & \vdots     & \multicolumn{5}{c||}{\vdots}                   & \multicolumn{5}{c|}{\vdots}        \\
\hline
17     & Pál Balázs &  8  &  5  &  6  &  10  &  3                    &  7  &  2  &  4  &  1  &  9         \\
\hline
\vdots & \vdots     & \multicolumn{5}{c||}{\vdots}                   & \multicolumn{5}{c|}{\vdots}        \\
\hline
\end{tabular}
\end{center}

\section*{Reducibility}
\textbf{Def.}\ \ \ An \textbf{internal vertex} is a node which is either part of a propagator loop, or have more than one connections. Internal vertices are marked with filled dots ($\bullet$) on the figures. Similarly, external vertices could be defined as the opposite of internal vertices, which are marked as empty dots ($\circ$) on the diagrams. \\
\textbf{Def.}\ \ \ An \textbf{internal edge/line} is an edge which connects two internal vertices. In contrast, external edges/lines are always connected to at least one external vertex. \\
\textbf{Def.}\ \ \ We call a diagram \textbf{reducible} which fall into two disjunct pieces if we cut one internal propagator line. \\
\textbf{Def.}\ \ \ Similarly, we call a diagram \textbf{irreducible} which does not fall into two disjunct pieces if we cut one internal propagator line. \\ \\
Using these definitions, we can easily identify the reducible and irreducible graphs:

\begin{center}
\begin{tabular}{|c|c|c|c||c|c|c|c|c|c|}
\hline
\multicolumn{4}{|c||}{Reducible graphs} & \multicolumn{6}{c|}{Irreducible graphs} \\
\hline
1 & 3 & 7 & 8 & 2 & 4 & 5 & 6 & 9 & 10                                            \\
\hline
\end{tabular}
\end{center}

\subsection*{Reducible graphs}
\begin{multicols}{4}

% GRAPH 1.
\begin{center}
\begin{tikzpicture}
  \begin{feynman}
    \vertex[empty dot] (a1){};
    \vertex[dot, right=0.8cm of a1] (a2){};
    \vertex[dot, right=0.8cm of a2] (a3){};
    \vertex[dot, right=0.8cm of a3] (a4){};
    \vertex[empty dot, right=0.8cm of a4] (a5){};
    
    \vertex[dot, above=0.8cm of a2] (b1){};
    
    \vertex[above=0.4cm of b1] (c1);
    \vertex[right=0.4cm of a3] (c2);
    
    \vertex[above right=0.4cm of a2] (cut1);
    \vertex[below=1.2cm of cut1] (cut2);
    \vertex[above right=0.8cm of cut1] (cut3);

    \diagram* {
      {[edges=fermion]
        (a5) -- (a4) -- (a3) -- (a2) -- (a1),
      },
      (a2) -- [dash pattern=on 4pt off 4pt] (b1),
      
      (cut1) -- [red, dash pattern=on 2pt off 2pt, line width=0.25mm] (cut2),
      (cut1) -- [red, dash pattern=on 2pt off 2pt, line width=0.25mm] (cut3),

    };
    \centerarc[fermion](c1)(0:360:0.4)
    \centerarc[dash pattern=on 4pt off 4pt](c2)(180:360:0.4)

  \end{feynman}
\end{tikzpicture}
\end{center}

% GRAPH 3.
\begin{center}
\begin{tikzpicture}
  \begin{feynman}
    \vertex[empty dot] (a1){};
    \vertex[dot, right=0.8cm of a1] (a2){};
    \vertex[dot, right=0.8cm of a2] (a3){};
    \vertex[dot, right=0.8cm of a3] (a4){};
    \vertex[empty dot, right=0.8cm of a4] (a5){};
    
    \vertex[dot, above=0.8cm of a4] (b1){};
    
    \vertex[above=0.4cm of b1] (c1);
    \vertex[right=0.4cm of a2] (c2);
    
    \vertex[above right=0.4cm of a3] (cut1);
    \vertex[below=1.2cm of cut1] (cut2);
	\vertex[above left=0.9cm of cut1] (cut3);

    \diagram* {
      {[edges=fermion]
        (a5) -- (a4) -- (a3) -- (a2) -- (a1),
      },
      (a4) -- [dash pattern=on 4pt off 4pt] (b1),
      
      (cut1) -- [red, dash pattern=on 2pt off 2pt, line width=0.25mm] (cut2),
      (cut1) -- [red, dash pattern=on 2pt off 2pt, line width=0.25mm] (cut3),

    };
    \centerarc[fermion](c1)(0:360:0.4)
    \centerarc[dash pattern=on 4pt off 4pt](c2)(180:360:0.4)

  \end{feynman}
\end{tikzpicture}
\end{center}

% GRAPH 7.
\begin{center}
\begin{tikzpicture}
  \begin{feynman}
    \vertex[empty dot] (a1){};
    \vertex[dot, right=0.8cm of a1] (a2){};
    \vertex[dot, right=1cm of a2] (a3){};
    \vertex[empty dot, right=0.8cm of a3] (a4){};
    
    \vertex[dot, above=0.8cm of a2] (b1){};
    \vertex[dot, above=0.8cm of a3] (b2){};

	\vertex[above=0.4cm of b1] (c1);
	\vertex[above=0.4cm of b2] (c2);
	
	\vertex[above right=0.7cm of a2] (cut1);
	\vertex[below=1.2cm of cut1] (cut2);

    \diagram* {
      {[edges=fermion]
        (a4) -- (a3) -- (a2) -- (a1),
      },
      (a2) -- [dash pattern=on 4pt off 4pt] (b1),
      (a3) -- [dash pattern=on 4pt off 4pt] (b2),
      
      (cut1) -- [red, dash pattern=on 2pt off 2pt, line width=0.25mm] (cut2),
    };
    \centerarc[fermion](c1)(0:360:0.4)
    \centerarc[fermion](c2)(0:360:0.4)

  \end{feynman}
\end{tikzpicture}
\end{center}

% GRAPH 8.
\begin{center}
\begin{tikzpicture}
  \begin{feynman}
    \vertex[empty dot] (a1){};
    \vertex[dot, right=0.8cm of a1] (a2){};
    \vertex[dot, right=0.8cm of a2] (a3){};
    \vertex[dot, right=0.8cm of a3] (a4){};
    \vertex[dot, right=0.8cm of a4] (a5){};
    \vertex[empty dot, right=0.8cm of a5] (a6){};
    
    \vertex[right=0.4cm of a2] (c1);
    \vertex[right=0.4cm of a4] (c2);
    
    \vertex[below right=0.6cm of a3] (cut1);
	\vertex[above=1.2cm of cut1] (cut2);

    \diagram* {
      {[edges=fermion]
        (a6) -- (a5) -- (a4) -- (a3) -- (a2) -- (a1),
      },      
      (cut1) -- [red, dash pattern=on 2pt off 2pt, line width=0.25mm] (cut2),

    };
    \centerarc[dash pattern=on 4pt off 4pt](c1)(0:180:0.4)
    \centerarc[dash pattern=on 4pt off 4pt](c2)(0:180:0.4)

  \end{feynman}
\end{tikzpicture}
\end{center}

\end{multicols}

\section*{Expressing the graphs}

\subsection*{Graph 1. --- Frequency space}

\begin{center}
\begin{tikzpicture}
  \begin{feynman}
    \vertex (a1);
    \vertex[right=1.5cm of a1] (a2);
    \vertex[right=1.5cm of a2] (a3);
    \vertex[right=1.5cm of a3] (a4);
    \vertex[right=1.5cm of a4] (a5);
    
    \vertex[above=1.5cm of a2] (b1);
    
    \vertex[above=0.75cm of b1] (c1);
    \vertex[right=0.75cm of a3] (c2);

    \diagram* {
      {[edges=fermion]
        (a5) -- (a4) -- (a3) -- (a2) -- (a1),
      },
      (a2) -- [dash pattern=on 4pt off 4pt] (b1),

    };
    \centerarc[fermion](c1)(0:360:0.75)
    \centerarc[dash pattern=on 4pt off 4pt](c2)(180:360:0.75)

  \end{feynman}
\end{tikzpicture}
\end{center}

\subsection*{Solution}
\newpage
\subsection*{Graph 2. \moment}

\begin{center}
\begin{tikzpicture}
  \begin{feynman}
    \vertex[empty dot] (a1){};
    \vertex[dot, right=1.5cm of a1] (a2){};
    \vertex[dot, right=1.5cm of a2] (a3){};
    \vertex[dot, right=1.5cm of a3] (a4){};
    \vertex[empty dot, right=1.5cm of a4] (a5){};
    
    \vertex[dot, above=1.5cm of a3] (b1){};
    
    \vertex[above=0.75cm of b1] (c1);

    \diagram* {
      {[edges=fermion]
        (a5) -- (a4) -- (a3) -- (a2) -- (a1),
      },
      (a3) -- [line width=1pt, dash pattern=on 4pt off 4pt] (b1),

    };
    \centerarc[fermion](c1)(0:360:0.75)
	\draw[dash pattern=on 4pt off 4pt, bend left=90] (a4) to (a2);

  \end{feynman}
\end{tikzpicture}
\end{center}

\subsection*{Solution}
\newpage
\subsection*{Graph 3. \coord}

\begin{center}
\begin{tikzpicture}
  \begin{feynman}
    \vertex[empty dot, label=below:{$X$}] (a1){};
    \vertex[dot, right=1.5cm of a1, label=above:{$X_{2}'$}] (a2){};
    \vertex[dot, right=1.5cm of a2, label=above:{$X_{2}$}] (a3){};
    \vertex[dot, right=1.5cm of a3, label=below:{$X_{1}$}] (a4){};
    \vertex[empty dot, right=1.5cm of a4, label=below:{$X'$}] (a5){};
    
    \vertex[dot, above=1.5cm of a4, label=below right:{$X_{1}'$}] (b1){};
    
    \vertex[above=0.75cm of b1] (c1);
    \vertex[right=0.75cm of a2] (c2);

    \diagram* {
      {[edges=fermion]
        (a5) -- (a4) -- (a3) -- (a2) -- (a1),
      },
      (a4) -- [dash pattern=on 4pt off 4pt] (b1),

    };
    \centerarc[fermion](c1)(0:360:0.75)
    \centerarc[dash pattern=on 4pt off 4pt](c2)(180:360:0.75)

  \end{feynman}
\end{tikzpicture}
\end{center}

\subsection*{Solution}
Fermion propagator lines are building connections between the points $\left( X' \to X_{1} \right)$, $\left( X_{2}' \to X \right)$, $\left( X_{1} \to X_{2} \right)$, $\left( X_{2} \to X_{2}' \right)$ and $\left( X_{1}' \to X_{1}' \right)$, where the last one is a fermion loop. Their contributions are:

\begin{align} \label{eq:19}
- \mathcal{G}_{0} \left( X_{i}, X_{j} \right)
&\to
\left[ - \mathcal{G}_{0} \left( X_{2}, X_{1} \right) \right]
*
\left[ - \mathcal{G}_{0} \left( X_{2}', X_{2} \right) \right]
*
\left[ - \mathcal{G}_{0} \left( X_{1}', X_{1}' \right) \right]
*
\left[ - \mathcal{G}_{0} \left( X_{1}, X' \right) \right]
*
\left[ - \mathcal{G}_{0} \left( X, X_{2}' \right) \right]
= \nonumber \\
&=
- \mathcal{G}_{0} \left( X_{2}, X_{1} \right)
*
\mathcal{G}_{0} \left( X_{2}', X_{2} \right)
*
\mathcal{G}_{0} \left( X_{1}', X_{1}' \right)
*
\mathcal{G}_{0} \left( X_{1}, X' \right)
*
\mathcal{G}_{0} \left( X, X_{2}' \right)
\end{align}
Fermion loops also contribute to the Green's function. Since there are only one of them its contribution is

\begin{equation} \label{eq:20}
\left( -1 \right)^{F} \to \left( -1 \right)^{1} = -1
\end{equation}
Interaction happens between $\left( X_{1}, X_{1}' \right)$ and $\left( X_{2}, X_{2}' \right)$. Their contributions are

\begin{equation} \label{eq:21}
- \frac{1}{\hbar} v \left( X_{i} X_{i}' \right)
\to
\left( - \frac{1}{\hbar} \right) v \left( X_{1} X_{1}' \right)
*
\left( - \frac{1}{\hbar} \right) v \left( X_{2} X_{2}' \right)
=
\frac{1}{\hbar^{2}} v \left( X_{1} X_{1}' \right) v \left( X_{2} X_{2}' \right)
\end{equation}
Putting them all together, we need to integrate over all the internal $X_{i}$ points to get the Green's function:

\begin{align} \label{eq:22}
\mathcal{G} \left( X, X' \right)
&=
\int \text{d}X_{1} \int \text{d}X_{2} \int \text{d}X_{1}' \int \text{d}X_{2}'
*
\left( -1 \right)
*
\frac{1}{\hbar^{2}} v \left( X_{1} X_{1}' \right) v \left( X_{2} X_{2}' \right)
\times \nonumber \\
&\times
\left[
- \mathcal{G}_{0} \left( X_{2}, X_{1} \right)
*
\mathcal{G}_{0} \left( X_{2}', X_{2} \right)
*
\mathcal{G}_{0} \left( X_{1}', X_{1}' \right)
*
\mathcal{G}_{0} \left( X_{1}, X' \right)
*
\mathcal{G}_{0} \left( X, X_{2}' \right)
\right]
\end{align}
Where the $(-1)$ term cancels out the minus sign at the $\mathcal{G}_{0}$ contributions. Thus the Green's function is the following:

\begin{align} \label{eq:23}
\mathcal{G} \left( X, X' \right)
&=
\int \text{d}X_{1} \int \text{d}X_{2} \int \text{d}X_{1}' \int \text{d}X_{2}'
*
\frac{1}{\hbar^{2}} v \left( X_{1} X_{1}' \right) v \left( X_{2} X_{2}' \right)
\times \nonumber \\
&\times
\mathcal{G}_{0} \left( X_{2}, X_{1} \right)
*
\mathcal{G}_{0} \left( X_{2}', X_{2} \right)
*
\mathcal{G}_{0} \left( X_{1}', X_{1}' \right)
*
\mathcal{G}_{0} \left( X_{1}, X' \right)
*
\mathcal{G}_{0} \left( X, X_{2}' \right)
\end{align}
\newpage
\subsection*{Graph 4. \moment}

\begin{center}
\begin{tikzpicture}
  \begin{feynman}
    \vertex[empty dot] (a1){};
    \vertex[dot, right=1.5cm of a1] (a2){};
    \vertex[dot, right=1.5cm of a2] (a3){};
    \vertex[empty dot, right=1.5cm of a3] (a4){};
    
    \vertex[dot, above=1.5cm of a2] (b1){};
    \vertex[dot, above=1.5cm of a3] (b2){};

	\vertex[right=0.75cm of b1] (c1);

    \diagram* {
      {[edges=fermion]
        (a4) -- (a3) -- (a2) -- (a1),
      },
      (a2) -- [dash pattern=on 4pt off 4pt] (b1),
      (a3) -- [dash pattern=on 4pt off 4pt] (b2),

    };
    \centerarc[fermion](c1)(0:180:0.75)
    \centerarc[fermion](c1)(180:360:0.75)

  \end{feynman}
\end{tikzpicture}
\end{center}

\subsection*{Solution}
\newpage
\subsection*{Graph 5. --- Coordinate space}

\begin{center}
\begin{tikzpicture}
  \begin{feynman}
    \vertex (a1);
    \vertex[right=1.5cm of a1] (a2);
    \vertex[right=1.5cm of a2] (a3);
    \vertex[right=1.5cm of a3] (a4);
    \vertex[right=1.5cm of a4] (a5);
    \vertex[right=1.5cm of a5] (a6);
    
    \vertex[right=0.75cm of a3] (c1);

    \diagram* {
      {[edges=fermion]
        (a6) -- (a5) -- (a4) -- (a3) -- (a2) -- (a1),
      },
    };
    \centerarc[dash pattern=on 4pt off 4pt](c1)(180:360:0.75)
	\draw[dash pattern=on 4pt off 4pt, bend right=90] (a5) to (a2);

  \end{feynman}
\end{tikzpicture}
\end{center}

\subsection*{Solution}
\newpage
\subsection*{Graph 6. --- Coordinate space}

\begin{center}
\begin{tikzpicture}
  \begin{feynman}
    \vertex (a1);
    \vertex[right=1.5cm of a1] (a2);
    \vertex[right=1.5cm of a2] (a3);
    
    \vertex[above=1.5cm of a2] (b1);
    \vertex[above=1.5cm of b1] (b2);
    \vertex[above=1.5cm of b2] (b3);

	\vertex[above=0.75cm of b1] (c1);
	\vertex[above=0.75cm of b3] (c2);

    \diagram* {
      {[edges=fermion]
        (a3) -- (a2) -- (a1),
      },
      (a2) -- [dash pattern=on 4pt off 4pt] (b1),
      (b2) -- [dash pattern=on 4pt off 4pt] (b3),
    };
    \centerarc[fermion](c1)(90:270:0.75)
    \centerarc[fermion](c1)(270:450:0.75)
    
    \centerarc[fermion](c2)(0:360:0.75)

  \end{feynman}
\end{tikzpicture}
\end{center}

\subsection*{Solution}
\newpage
\subsection*{Graph 7. \moment}

\begin{center}
\begin{tikzpicture}
  \begin{feynman}
    \vertex[empty dot, label=below:{$x$}] (a1){};
    \vertex[dot, right=2cm of a1, label=below:{$x_{1}$}] (a2){};
    \vertex[dot, right=2.2cm of a2, label=below:{$x_{2}$}] (a3){};
    \vertex[empty dot, right=2cm of a3, label=below:{$x'$}] (a4){};
    
    \vertex[dot, above=2cm of a2, label=above:{$x_{1}'$}] (b1){};
    \vertex[dot, above=2cm of a3, label=above:{$x_{2}'$}] (b2){};
    
    \vertex[above=2cm of b1] (b3);
    \vertex[above=2cm of b2] (b4);

	\vertex[above=1cm of b1] (c1);
	\vertex[above=1cm of b2] (c2);

    \diagram* {
      {[edges=fermion]
        (a4) -- [momentum=$\boldsymbol{k}; i \nu_{n}$] (a3),
        (a3) -- [momentum=$\boldsymbol{q}; i \nu_{m}$] (a2),
        (a2) -- [momentum=$\boldsymbol{k}; i \nu_{n}$] (a1),
      },
      (a2) -- [rmomentum=$\boldsymbol{0}; 0$, dash pattern=on 4pt off 4pt] (b1),
      (a3) -- [rmomentum'=$\boldsymbol{0}; 0$, dash pattern=on 4pt off 4pt] (b2),
      (b1) -- [half right, looseness=1.6, in=-95, out=-85] (b3),
      (b3) -- [momentum'=$\boldsymbol{k}'; i \nu_{n}'$, half right, looseness=1.6, in=-95, out=-85, fermion] (b1),
      (b2) -- [momentum'=$\boldsymbol{k}''; i \nu_{n}''$, half right, looseness=1.6, in=-95, out=-85] (b4),
      (b4) -- [half right, looseness=1.6, in=-95, out=-85, fermion] (b2),
    };
    %\centerarc[fermion](c1)(0:360:1)
    %\centerarc[fermion](c2)(0:360:1)

  \end{feynman}
\end{tikzpicture}
\end{center}

\subsection*{Solution}
\newpage
\subsection*{Graph 8. \coord}

\begin{center}
\begin{tikzpicture}
  \begin{feynman}
    \vertex[empty dot, label=below:{$X$}] (a1){};
    \vertex[dot, right=1.5cm of a1, label=below:{$X_{1}'$}] (a2){};
    \vertex[dot, right=1.5cm of a2, label=below:{$X_{1}$}] (a3){};
    \vertex[dot, right=1.5cm of a3, label=below:{$X_{2}'$}] (a4){};
    \vertex[dot, right=1.5cm of a4, label=below:{$X_{2}$}] (a5){};
    \vertex[empty dot, right=1.5cm of a5, label=below:{$X'$}] (a6){};
    
    \vertex[right=0.75cm of a2] (c1);
    \vertex[right=0.75cm of a4] (c2);

    \diagram* {
      {[edges=fermion]
        (a6) -- (a5) -- (a4) -- (a3) -- (a2) -- (a1),
      },
    };
    \centerarc[dash pattern=on 4pt off 4pt](c1)(0:180:0.75)
    \centerarc[dash pattern=on 4pt off 4pt](c2)(0:180:0.75)

  \end{feynman}
\end{tikzpicture}
\end{center}

\subsection*{Solution}
Fermion propagator lines are building connections between the points $\left( X' \to X_{2} \right)$, $\left( X_{2} \to X_{2}' \right)$, $\left( X_{2}' \to X_{1} \right)$, $\left( X_{1} \to X_{1}' \right)$ and $\left( X_{1}' \to X \right)$. Their contributions are:

\begin{align} \label{eq:48}
- \mathcal{G}_{0} \left( X_{i}, X_{j} \right)
&\to
\left[ - \mathcal{G}_{0} \left( X_{2}, X' \right) \right]
*
\left[ - \mathcal{G}_{0} \left( X_{2}', X_{2} \right) \right]
*
\left[ - \mathcal{G}_{0} \left( X_{1}, X_{2}' \right) \right]
*
\left[ - \mathcal{G}_{0} \left( X_{1}', X_{1} \right) \right]
*
\left[ - \mathcal{G}_{0} \left( X, X_{1}' \right) \right]
= \nonumber \\
&=
- \mathcal{G}_{0} \left( X_{2}, X' \right)
*
\mathcal{G}_{0} \left( X_{2}', X_{2} \right)
*
\mathcal{G}_{0} \left( X_{1}, X_{2}' \right)
*
\mathcal{G}_{0} \left( X_{1}', X_{1} \right)
*
\mathcal{G}_{0} \left( X, X_{1}' \right)
\end{align}
Interaction happens between $\left( X_{1}, X_{1}' \right)$ and $\left( X_{2}, X_{2}' \right)$. Their contributions are

\begin{equation} \label{eq:49}
- \frac{1}{\hbar} v \left( X_{i} X_{i}' \right)
\to
\left( - \frac{1}{\hbar} \right) v \left( X_{1} X_{1}' \right)
*
\left( - \frac{1}{\hbar} \right) v \left( X_{2} X_{2}' \right)
=
\frac{1}{\hbar^{2}} v \left( X_{1} X_{1}' \right) v \left( X_{2} X_{2}' \right)
\end{equation}
Putting them all together, we need to integrate over all the internal $X_{i}$ points to get the final form for the integral of the Green's function:

\begin{align} \label{eq:50}
\mathcal{G} \left( X, X' \right)
&=
\int \text{d}X_{1} \int \text{d}X_{2} \int \text{d}X_{1}' \int \text{d}X_{2}'
*
\frac{1}{\hbar^{2}} v \left( X_{1} X_{1}' \right) v \left( X_{2} X_{2}' \right)
\times \nonumber \\
&\times
\left[
- \mathcal{G}_{0} \left( X_{2}, X' \right)
*
\mathcal{G}_{0} \left( X_{2}', X_{2} \right)
*
\mathcal{G}_{0} \left( X_{1}, X_{2}' \right)
*
\mathcal{G}_{0} \left( X_{1}', X_{1} \right)
*
\mathcal{G}_{0} \left( X, X_{1}' \right)
\right]
\end{align}
\newpage
\subsection*{Graph 9. \moment}

\begin{center}
\begin{tikzpicture}
  \begin{feynman}
    \vertex[empty dot] (a1){};
    \vertex[dot, right=1.5cm of a1] (a2){};
    \vertex[empty dot, right=1.5cm of a2] (a3){};
    
    \vertex[dot, above=1.5cm of a2] (b1){};

    \vertex[above=0.75cm of b1] (c1);
    
    \vertex[dot, right=0.75cm of c1] (b2){};
    \vertex[dot, left=0.75cm of c1] (b3){};

    \diagram* {
      {[edges=fermion]
        (a3) -- (a2) -- (a1),
      },
      (a2) -- [dash pattern=on 4pt off 4pt] (b1),
      (b2) -- [dash pattern=on 4pt off 4pt] (b3),

    };
    \centerarc[fermion](c1)(0:180:0.75)
    \centerarc[fermion](c1)(180:270:0.75)
    \centerarc[fermion](c1)(270:360:0.75)

  \end{feynman}
\end{tikzpicture}
\end{center}

\subsection*{Solution}
\newpage
\subsection*{Graph 10. \coord}

\begin{center}
\begin{tikzpicture}
  \begin{feynman}
    \vertex[empty dot] (a1){};
    \vertex[dot, right=1.5cm of a1] (a2){};
    \vertex[dot, right=1.5cm of a2] (a3){};
    \vertex[dot, right=1.5cm of a3] (a4){};
    \vertex[dot, right=1.5cm of a4] (a5){};
    \vertex[empty dot, right=1.5cm of a5] (a6){};

    \diagram* {
      {[edges=fermion]
        (a6) -- (a5) -- (a4) -- (a3) -- (a2) -- (a1),
      },
    };
    \draw[dash pattern=on 4pt off 4pt, bend right=90] (a5) to (a3);
    \draw[dash pattern=on 4pt off 4pt, bend left=90] (a4) to (a2);
    %\centerarc[dash pattern=on 4pt off 4pt](a3)(180:360:1.5)
    %\centerarc[dash pattern=on 4pt off 4pt](a4)(0:180:1.5)

  \end{feynman}
\end{tikzpicture}
\end{center}

\subsection*{Solution}